%!TEX root = ../Grunddatei.tex
\section{Auswertung}

Im ersten Versuch zu den Kenngrößen eines aktiven Zweipols wurde zunächst die verschiedenen Schaltungen aufgebaut und
anschließend die Größen $U_q$ und $I_k$ mittels Messungen bestimmt. Aus diesen
Größen sind dann der Innenwiderstand $R_i$ sowie die einzelnen Wirkungsgrade als auch die
abgegebene Leistungen bestimmt worden.\\
Im zweiten Versuch wurden zunächst die einzelnen Widerstände gemessen als auch den
Ersatzinnenwiderstand bei überbrückter Quellspannung. Dieser stimmte mit dem
berechneten Innenwiderstand annähernd übereinstimmen.\\
Im dritten Versuch konnten wir erfolgreich die einzelnen Widerstände bestimmen und den Zweigstrom direkt messen aber auch durch Überlagerung berechnen. Die berechneten Werte stimmten mit den gemessenen Werten überein, so dass hier besonders die theoretischen Überlegungen der Vorbereitung verifiziert werden konnten.

In diesem Praktikum haben wir uns näher mit dem Zusammenhang von Strom und Spannung im
elektrischen Grundstromkreis beschäftigt und vertraut gemacht. Durch anfängliche Schwierigkeiten
haben wir uns noch intensiver mit den einzelnen Versuchen auseinandergesetzt und haben somit
unser gesetztes Ziel erfüllt und erreicht.
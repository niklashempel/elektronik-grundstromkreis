\documentclass[
   12pt, % Schriftgröße
   DIV=calc,
   ngerman, % für Umlaute, Silbentrennung etc.
   a4paper, % Papierformat
   oneside, % einseitiges Dokument
   titlepage, % es wird eine Titelseite verwendet
   parskip=half, % Abstand zwischen Absätzen (halbe Zeile)
%   headings=normal, % Grˆfle der Überschriften verkleinern
   listof=totoc, % Verzeichnisse im Inhaltsverzeichnis aufführen
   bibliography=totoc, % Literaturverzeichnis im Inhaltsverzeichnis aufführen
   index=totoc, % Index im Inhaltsverzeichnis aufführen
%   captions=nooneline, % Beschriftung von Tabellen unterhalb ausgeben
   final, % Status des Dokuments (final/draft)
   numbers=noenddot
]{scrartcl}

%Abstände ändern
%\renewcommand*{\chapterheadstartvskip}{\vspace{-1\baselineskip}} %Abstand vor Section
\setlength{\headsep}{30pt} %Abstand Kopfzeile Textkörper

\usepackage{graphicx}

% !TEX root = ../Grunddatei.tex
%Wenn die Texte Umlaute oder ein Esszet enthalten, muss der folgende
%Befehl bereits an dieser Stelle aktiviert werden:
%\usepackage[latin1]{inputenc}

\newcommand{\art}{Art der Arbeit}
\newcommand{\titel}{Titel der Arbeit}
\newcommand{\untertitel}{Untertitel}
\newcommand{\fachgebiet}{Bla}
\newcommand{\autor}{Name des Autors}
\newcommand{\studienbereich}{Name des Studienbereichs}
\newcommand{\matrikelnr}{Matrikelnummer des Autors}
\newcommand{\erstgutachter}{Name des Erstgutachters}
\newcommand{\zweitgutachter}{Name des Zweitgutachters}
\newcommand{\jahr}{20XX}
\newcommand{\ort}{HTW Dresden}
\newcommand{\logolang}{Bilder/HTWD_logo_CMYK_horizontal_color.pdf}
\newcommand{\logo}{Bilder/HTWD_logo_CMYK_wordmark_color.pdf}
 %Informationen über das Dokument

% !TEX root = ../Grunddatei.tex
\usepackage[utf8]{inputenc}
\usepackage{amsmath}
\usepackage{amsfonts}
\usepackage{amssymb}
\usepackage{graphicx}
\usepackage{subcaption}
\usepackage{ngerman}
\usepackage{here}
\usepackage{color}
\usepackage{textcomp}
\usepackage{pdfpages}
\usepackage[paper=a4paper,left=25mm,right=25mm,top=25mm,bottom=25mm]{geometry}


%Zeilenabstände und Seitenränder
\usepackage{setspace}
\usepackage{geometry}

\usepackage[
    automark, %Kapitelangaben in Kopfzeile automatisch erstellen
    headsepline, %Trennlinie unter Kopfzeile
    ilines %Trennlinie linksbündig ausrichten
]{scrlayer-scrpage}

%fortlaufendes Durchnummerieren der Fußnoten
\usepackage{chngcntr}

%wichtig für korrekte Zitierweise
\usepackage[numbers,square]{natbib}

%für lange Tabellen -----------------------------------------------------------
\usepackage{longtable}
\usepackage{colortbl}
\usepackage{array}
\usepackage{ragged2e}
\usepackage{pdflscape}

%Abkürzungsbibliothek
\usepackage[printonlyused]{acronym}
%Glossar
\usepackage{glossaries}

%Code
%Siehe auch: http://ctan.ebinger.cc/tex-archive/macros/latex/contrib/listings/lstdrvrs.pdf
\usepackage{listings}
\usepackage[scaled]{helvet}
\renewcommand*\familydefault{\sfdefault}
\usepackage[T1]{fontenc}
\lstset{
    basicstyle=\scriptsize\ttfamily, 	% Default font
    numbers=left,              			% Location of line numbers
    numberstyle=\tiny,          		% Style of line numbers
    % stepnumber=2,              		% Margin between line numbers
    numbersep=5pt,             			% Margin between line numbers and text
    tabsize=2,                  		% Size of tabs
    extendedchars=true,
    breaklines=true,            		% Lines will be wrapped
    keywordstyle=\color{red},
    frame=single,
    % keywordstyle=[1]\textbf,
    % keywordstyle=[2]\textbf,
    % keywordstyle=[3]\textbf,
    % keywordstyle=[4]\textbf,   \sqrt{\sqrt{}}
    stringstyle=\color{white}\ttfamily, % Color of strings
    showspaces=false,
    showtabs=false,
    xleftmargin=17pt,
    %framexleftmargin=17pt,
    %framexrightmargin=5pt,
    %framexbottommargin=4pt,
    % backgroundcolor=\color{lightgray},
    showstringspaces=false,
    numberbychapter=true
}

\usepackage{chngcntr}
\counterwithin{figure}{section}

\usepackage[perpage]{footmisc} %benötigte Packages
 
\makeindex
%\makeglossaries

% !TEX root = Grunddatei.tex

% Times Roman Schriftart
\setkomafont{disposition}{\normalcolor\bfseries}

% Zeilenabstand 1,5 Zeilen
\onehalfspacing

\pagestyle{scrheadings} %Kopf- und Fuflzeilen

\renewcommand{\headfont}{\normalfont} %Schriftform der Kopfzeile

% Kopfzeile
\ihead{\textit{\headmark}}
\chead{\hspace{11cm} \includegraphics[scale=0.15]{\logo}}
\ohead{}

\setlength{\headheight}{21mm} % Höhe der Kopfzeile

\setheadwidth[0pt]{textwithmarginpar} %Kopfzeile über den Text hinaus verbreitern

\setheadsepline[text]{0.4pt} % Trennlinie unter Kopfzeile

% Fußzeile
\ifoot{}
\cfoot{}
\ofoot{\pagemark} %Kopf- und Fußzeilen, Seitenränder..

%Hier beginnt das eigentliche Dokument
\begin{document}

\setcounter{secnumdepth}{3}
\setcounter{tocdepth}{3}

%% !TEX root = ../Grunddatei.tex
\thispagestyle{plain}
\begin{titlepage}
	\begin{center}
		\vspace*{2.5cm}
		\includegraphics[width=0.5\linewidth]{\logolang}\\[1.25cm]
		\Large{\textbf{\art}}\\[1.5ex]
		\LARGE{\textbf{\titel}}\\[3.5ex]
		

		\normalsize
		\begin{tabular}{p{5.4cm}p{7cm}}\\
			vorgelegt von:  & \quad \autor\\[1.2ex]
			Studienbereich: & \quad \studienbereich\\[1.2ex]
			Ort: & \quad \ort\\[1.2ex]
			Matrikelnummer: & \quad \matrikelnr\\[1.2ex]
			Erstgutachter:  & \quad \erstgutachter\\[1.2ex]
			Zweitgutachter: & \quad \zweitgutachter\\[8ex]
			Abgabedatum:    & \quad ...
		\end{tabular}
		
		\copyright\ \jahr\\[9ex]
	\end{center}
	
	%\singlespacing
	%\small
	%\noindent Dieses Werk einschliefllich seiner Teile ist \textbf{urheberrechtlich geschützt}. Jede Verwertung außerhalb der engen Grenzen des Urheberrechtgesetzes ist ohne Zustimmung des Autors unzulässig und strafbar. 
	%Das gilt insbesondere für Vervielfältigungen, Übersetzungen, Mikroverfilmungen sowie die  Einspeicherung und Verarbeitung in elektronischen Systemen.
	
\end{titlepage} %Deckblatt

% !TEX root = Grunddatei.tex
 %Falls ein Abstract eingefügt werden soll

\pagenumbering{gobble}

%%!TEX root =../Grunddatei.tex
\addsec{Danksagung}
Hier würde die Danksagung stehen %Danksagung

\pagenumbering{Roman}

\tableofcontents %Inhaltsverzeichnis

%% !TEX root = Grunddatei.tex
\addsec{Abkürzungsverzeichnis}
\begin{acronym}[laaaaaaaaaaang] 
%In der Eckigen Klammer muss das längste Akronym stehen. 
%Es wird dann die gesamte weitere Tabelle danach ausgerichtet.
\end{acronym} %Abkürzungsverzeichnis

%% !TEX root = Grunddatei.tex
\addsec{Glossar}
\newglossaryentry{latex}
{
	name=latex,
	description={Is a mark up language specially suited
			for scientific documents}
}



\printglossaries %Glossar

\newpage
\listoffigures % Abbildungsverzeichnis
\newpage
\listoftables % Tabellenverzeichnis

\clearpage

% 1.5-fachen Zeilenabstand
\onehalfspacing

% Text includes

\pagenumbering{arabic}
% !TEX root = Grunddatei.tex
%Hier können die einzelnen Kapitel hinzugefügt werden
%Sie müssen in den entsprechenden .TEX-Dateien vorliegen. 
%Die Dateinamen können natürlich angepasst werden.

% !TEX root = ../Grunddatei.tex
\section{Einleitung}



 
\newpage
% !TEX root = ../Grunddatei.tex
\section{Grundlagen}
 
\newpage
% !TEX root = ../Grunddatei.tex
\section{Material und Methoden}
Es wird aus einem Buch zitiert: Lorem ipsum\cite{Krom}
\newpage
% !TEX root = ../Grunddatei.tex
\section{Ergebnisse und Diskussion}

\newpage
% !TEX root = ../Grunddatei.tex
\section{Zusammenfassung und Ausblick}

% Literaturverzeichnis 
\bibliography{Bibliographie} % Aufruf: bibtex Bibliographie.bib
\bibliographystyle{natdin} % DIN-Stil des Literaturverzeichnisses

% Anhang 
\begin{appendix}
   \clearpage
   \pagenumbering{arabic}
   %!TEX root = ../Grunddatei.tex
\section{Anhang}

\end{appendix}

% Selbständigkeitserklärung
% !TEX root = Grunddatei.tex
\pagenumbering{gobble}
\addsec{Erklärung}
Der Verfasser erklärt, dass er die vorliegende Arbeit selbständig, ohne fremde Hilfe und ohne Benutzung anderer als die angegebenen Hilfsmittel angefertigt hat. Die aus fremden Quellen (einschließlich elektronischer Quellen) direkt oder indirekt übernommenen Gedanken sind ausnahmslos als solche kenntlich gemacht. Die Arbeit ist in gleicher oder ähnlicher Form oder auszugsweise im Rahmen einer anderen Prüfung noch nicht vorgelegt worden.Zudem bestätigt der Verfasser, dass er den Lehrstuhlleitfaden in der jeweiligen geltenden Fassung gelesen und verstanden hat und sich daher über die gestellten Anforderungen und Bewertungsmaßstäbe im Klaren ist. \\[4cm]
\noindent\rule{5cm}{.4pt}\hfill\rule{5cm}{.4pt}\par 
\noindent Datum, Ort \hfill \autor \\
 %Erklaerung.tex

% Ende des Dokuments
\end{document}
%------------------------------------------